\hypertarget{group___c_m_s_i_s___m_i_s_r_a___exceptions}{}\section{C\+M\+S\+IS M\+I\+S\+R\+A-\/C\+:2004 Compliance Exceptions}
\label{group___c_m_s_i_s___m_i_s_r_a___exceptions}\index{C\+M\+S\+I\+S M\+I\+S\+R\+A-\/\+C\+:2004 Compliance Exceptions@{C\+M\+S\+I\+S M\+I\+S\+R\+A-\/\+C\+:2004 Compliance Exceptions}}
C\+M\+S\+IS violates following M\+I\+S\+R\+A-\/\+C2004 Rules\+:


\begin{DoxyItemize}
\item Violates M\+I\+S\+RA 2004 Required Rule 8.\+5, object/function definition in header file.~\newline
 Function definitions in header files are used to allow \textquotesingle{}inlining\textquotesingle{}.
\item Violates M\+I\+S\+RA 2004 Required Rule 18.\+4, declaration of union type or object of union type\+: \textquotesingle{}\{...\}\textquotesingle{}.~\newline
 Unions are used for effective representation of core registers.
\item Violates M\+I\+S\+RA 2004 Advisory Rule 19.\+7, Function-\/like macro defined.~\newline
 Function-\/like macros are used to allow more efficient code. 
\end{DoxyItemize}